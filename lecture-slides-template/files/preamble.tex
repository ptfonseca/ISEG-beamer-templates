
% Document class
%--------------------------------------------------------------  
\documentclass{beamer} 

% Theme and subtheme
%--------------------------------------------------------------
\usetheme{Berlin}                             
\usecolortheme{beaver}       

% Disable the navigation symbols
%--------------------------------------------------------------        
\beamertemplatenavigationsymbolsempty  

% Packages  and options
%--------------------------------------------------------------  
\usepackage{fancyvrb}
\usepackage{textpos}                       
\usepackage{wasysym}                     
\usepackage{csquotes}                  
\usepackage{wrapfig}                       
\usepackage{microtype}                   
\usepackage[labelfont=bf]{caption}   
\usepackage{float}
\usepackage{hyperref}
\usepackage{amsmath}
\newcommand{\dd}[1]{\mathrm{d}#1}
\usepackage{amsfonts}
\usepackage{amssymb}
\usepackage{amsbsy}
\usepackage{adjustbox}
\setbeamertemplate{caption}[numbered] 
\usepackage{subcaption}
\usepackage[none]{hyphenat} 
\usepackage{bm} 
\usepackage{tabularx}  
\usepackage{tikz}
\PassOptionsToPackage{fancyvrb, newverbs, dvipsnames, table}{xcolor}

% ISEG name and logo
%--------------------------------------------------------------  
\titlegraphic{\includegraphics[width=1.5cm]{figures/logo}}
\institute[ISEG Lisbon School of Economics \& Management]{\textbf {ISEG Lisbon School of Economics \& Management}}   

% ISEG logo on the frame title
%--------------------------------------------------------------  
\addtobeamertemplate{frametitle}{}{%
\begin{textblock*}{100mm}(.99\textwidth,-.9cm)
\includegraphics[scale=.015]{figures/logo}
\end{textblock*}}

% Create the Maroon color
%--------------------------------------------------------------  
\definecolor{Maroon}{cmyk}{0, 0.87, 0.68, 0.32}

% ISEG colors everywhere
%--------------------------------------------------------------  
\setbeamercolor{author in head/foot}{fg=Maroon}     
\setbeamercolor{institute in head/foot}{fg=Maroon}  
\setbeamercolor{frametitle}{fg=Maroon, bg=black!9} 
\setbeamercolor{caption name}{fg=Maroon, bg=black!9}
\setbeamercolor*{title}{fg=white, bg=Maroon}          
\setbeamercolor{itemlist item}{fg=Maroon}               
\setbeamercolor{section number projected}{bg=Maroon,fg=white}
\setbeamercolor{subsection number projected}{bg=Maroon,fg=white}
\setbeamercolor{subsubsection number projected}{bg=Maroon,fg=white}
\setbeamercolor{block title}{fg=Maroon}
\setbeamercolor{local structure}{fg=Maroon} 

% ISEG colors on the Itemize bullets 
% Different bullet shapes for each itemize level
%--------------------------------------------------------------  
\setbeamertemplate{itemize item}{\color{Maroon}\newmoon}
\setbeamertemplate{itemize subitem}{\color{Maroon}$\blacktriangleright$}
\setbeamertemplate{itemize subsubitem}{\color{Maroon}$\blacksquare$}

% TOC colors and bullet shapes
%--------------------------------------------------------------  
\setbeamertemplate{section in toc}[circle]
\setbeamercolor{section number projected}{bg=Maroon, fg=white}
\setbeamertemplate{subsection in toc}{%
\leavevmode\leftskip=5.65ex%
\llap{\raisebox{0.2ex}{\textcolor{structure}{\color{Maroon}$\blacktriangleright$}}\kern1ex}%
\inserttocsubsection\par%  
}

% New environment for definitions/theorems with ISEG colors
%------------------------------------------------------------
\makeatletter
\def\th@mystyle{%
    \normalfont % body font
    \setbeamercolor{block title example}{bg=BrickRed,fg=white}
    \setbeamercolor{block body example}{bg=black!9,fg=black}
    \def\inserttheoremblockenv{exampleblock}
  }
\makeatother
\theoremstyle{mystyle}
\newtheorem*{defi}{Definition}

\setbeamertemplate{defis}[numbered]

% New command to display two-part functions 
%--------------------------------------------------------------  
\newcommand{\twopartdef}[4] 
{
	\left\{
		\begin{array}{ll}
			#1 & \mbox{if } #2 
			\\ \\
			#3 & \mbox{if } #4
		\end{array}
	\right.
}

% New enviroment without headline on the frame header
%--------------------------------------------------------------  
\makeatletter 
\newenvironment{noheadline}{
\setbeamertemplate{headline}{}
\addtobeamertemplate{frametitle}{\vspace*{-0.9\baselineskip}}{}
}{}
\makeatother

% To highlight/color specific rows in tables
%--------------------------------------------------------------  
\usepackage[beamer,customcolors]{hf-tikz}

\tikzset{hl/.style={
    set fill color=red!80!black!40,
    set border color=red!80!black,
  },
}

% Create the color for the verbatim background
%--------------------------------------------------------------  
\definecolor{cverbbg}{gray}{0.93}

% New environment for verbatin with background color
%--------------------------------------------------------------  
\newenvironment{cverbatim}
 {\SaveVerbatim{cverb}}
 {\endSaveVerbatim
  \flushleft\fboxrule=0pt\fboxsep=.5em
  \colorbox{cverbbg}{%
    \makebox[\dimexpr\linewidth-2\fboxsep][l]{\BUseVerbatim{cverb}}%
  }
  \endflushleft
}

% Get rid of the "references" section in the header
%------------------------------------------------------------
\makeatletter
\let\beamer@writeslidentry@miniframeson=\beamer@writeslidentry%
\def\beamer@writeslidentry@miniframesoff{%
  \expandafter\beamer@ifempty\expandafter{\beamer@framestartpage}{}% does not happen normally
  {%else
    % removed \addtocontents commands
    \clearpage\beamer@notesactions%
  }
}
\newcommand*{\miniframeson}{\let\beamer@writeslidentry=\beamer@writeslidentry@miniframeson}
\newcommand*{\miniframesoff}{\let\beamer@writeslidentry=\beamer@writeslidentry@miniframesoff}
\makeatother

% Add vertical space between the items of the TOC
%------------------------------------------------------------
\makeatletter
\patchcmd{\beamer@sectionintoc}
  {\vfill}
  {\vskip\itemsep}
  {}
  {}
\makeatother

% Bibliography packages and options
%------------------------------------------------------------
\usepackage[
	style=authoryear,
 	bibencoding=utf8, 
 	minnames=1, 
 	maxnames=3,
	maxbibnames=3, 
	backref=false, 
	natbib=true, 
	dashed=false, 
	terseinits=true, 
	giveninits=true, 
	uniquename=false, 
	uniquelist=true, 
	labeldateparts=true,
	doi=false, 
	isbn=false, 
	eprint = false,        % I disabled this. check if its appropriate
	url = false,             % I disabled this. check if its appropriate
natbib=true, backend=biber]{biblatex}

\DefineBibliographyStrings{english}{
    backrefpage = {Cited on page},
    backrefpages = {Cited on pages},
}

% Change the default bibliography formatting to be more "statistical"
%------------------------------------------------------------
\DeclareFieldFormat{url}{\url{#1}}
\DeclareFieldFormat[article]{pages}{#1}
\DeclareFieldFormat[inproceedings]{pages}{\lowercase{pp.}#1}
\DeclareFieldFormat[incollection]{pages}{\lowercase{pp.}#1}
\DeclareFieldFormat[article]{volume}{\mkbibbold{#1}}
\DeclareFieldFormat[article]{number}{\mkbibparens{#1}}
\DeclareFieldFormat[article]{title}{\MakeCapital{#1}}
\DeclareFieldFormat[article]{url}{}
\DeclareFieldFormat[book]{url}{}
\DeclareFieldFormat[inbook]{url}{}
\DeclareFieldFormat[thesis]{url}{}
\DeclareFieldFormat[thesis]{url}{}
\DeclareFieldFormat[incollection]{url}{}
\DeclareFieldFormat[inproceedings]{url}{}
\DeclareFieldFormat[inproceedings]{title}{#1}
\DeclareFieldFormat{shorthandwidth}{#1}
\DeclareFieldFormat[thesis]{citetitle}{#1}
\DeclareFieldFormat[thesis]{title}{#1} 

% No dot before number of articles
%------------------------------------------------------------
\usepackage{xpatch}
\xpatchbibmacro{volume+number+eid}{\setunit*{\adddot}}{}{}{}

% Remove In: for an article.
%------------------------------------------------------------
\renewbibmacro{in:}{%
  \ifentrytype{article}{}{%
  \printtext{\bibstring{in}\intitlepunct}}}
  
% Get rid of notes in citations
%------------------------------------------------------------
\AtEveryBibitem{\clearfield{note}}

% Get rid of months in citations
%------------------------------------------------------------
\AtEveryBibitem{\clearfield{month}}
\AtEveryCitekey{\clearfield{month}}
\raggedbottom

% Font for references
%------------------------------------------------------------
\renewcommand\bibfont{\scriptsize}

% If you have more than one page of references, you want to tell beamer
% to put the continuation section label from the second slide onwards
%------------------------------------------------------------
\setbeamertemplate{frametitle continuation}[from second]

% Get rid of all the colours
%------------------------------------------------------------
\setbeamercolor*{bibliography entry title}{fg=black}
\setbeamercolor*{bibliography entry author}{fg=black}
\setbeamercolor*{bibliography entry location}{fg=black}
\setbeamercolor*{bibliography entry note}{fg=black}

% No icon in references listing
%------------------------------------------------------------
\setbeamertemplate{bibliography item}{}

% path to the .bib file
%------------------------------------------------------------
\addbibresource{files/references.bib}

