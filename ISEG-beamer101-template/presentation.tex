
% Preamble
%--------------------------------------------------------------

% Document class
%--------------------------------------------------------------  
\documentclass{beamer} 

% Theme and subtheme
%--------------------------------------------------------------
\usetheme{Berlin}                             
\usecolortheme{beaver}       

% Disable the navigation symbols
%--------------------------------------------------------------        
\beamertemplatenavigationsymbolsempty  

% Packages  and options
%--------------------------------------------------------------  
\usepackage{fancyvrb}
\usepackage{textpos}                       
\usepackage{wasysym}                     
\usepackage{csquotes}                  
\usepackage{wrapfig}                       
\usepackage{microtype}                   
\usepackage[labelfont=bf]{caption}   
\usepackage{float}
\usepackage{hyperref}
\usepackage{amsmath}
\newcommand{\dd}[1]{\mathrm{d}#1}
\usepackage{amsfonts}
\usepackage{amssymb}
\usepackage{amsbsy}
\usepackage{adjustbox}
\setbeamertemplate{caption}[numbered] 
\usepackage{subcaption}
\usepackage[none]{hyphenat} 
\usepackage{bm} 
\usepackage{tabularx}  
\usepackage{tikz}
\PassOptionsToPackage{fancyvrb, newverbs, dvipsnames, table}{xcolor}

% ISEG name and logo
%--------------------------------------------------------------  
\titlegraphic{\includegraphics[width=1.5cm]{figures/logo}}
\institute[ISEG Lisbon School of Economics \& Management]{\textbf {ISEG Lisbon School of Economics \& Management}}   

% ISEG logo on the frame title
%--------------------------------------------------------------  
\addtobeamertemplate{frametitle}{}{%
\begin{textblock*}{100mm}(.99\textwidth,-.9cm)
\includegraphics[scale=.015]{figures/logo}
\end{textblock*}}

% Create the Maroon color
%--------------------------------------------------------------  
\definecolor{Maroon}{cmyk}{0, 0.87, 0.68, 0.32}

% ISEG colors everywhere
%--------------------------------------------------------------  
\setbeamercolor{author in head/foot}{fg=Maroon}     
\setbeamercolor{institute in head/foot}{fg=Maroon}  
\setbeamercolor{frametitle}{fg=Maroon, bg=black!9} 
\setbeamercolor{caption name}{fg=Maroon, bg=black!9}
\setbeamercolor*{title}{fg=white, bg=Maroon}          
\setbeamercolor{itemlist item}{fg=Maroon}               
\setbeamercolor{section number projected}{bg=Maroon,fg=white}
\setbeamercolor{subsection number projected}{bg=Maroon,fg=white}
\setbeamercolor{subsubsection number projected}{bg=Maroon,fg=white}
\setbeamercolor{block title}{fg=Maroon}
\setbeamercolor{local structure}{fg=Maroon} 

% ISEG colors on the Itemize bullets 
% Different bullet shapes for each itemize level
%--------------------------------------------------------------  
\setbeamertemplate{itemize item}{\color{Maroon}\newmoon}
\setbeamertemplate{itemize subitem}{\color{Maroon}$\blacktriangleright$}
\setbeamertemplate{itemize subsubitem}{\color{Maroon}$\blacksquare$}

% TOC colors and bullet shapes
%--------------------------------------------------------------  
\setbeamertemplate{section in toc}[circle]
\setbeamercolor{section number projected}{bg=Maroon, fg=white}
\setbeamertemplate{subsection in toc}{%
\leavevmode\leftskip=5.65ex%
\llap{\raisebox{0.2ex}{\textcolor{structure}{\color{Maroon}$\blacktriangleright$}}\kern1ex}%
\inserttocsubsection\par%  
}

% New environment for definitions/theorems with ISEG colors
%------------------------------------------------------------
\makeatletter
\def\th@mystyle{%
    \normalfont % body font
    \setbeamercolor{block title example}{bg=BrickRed,fg=white}
    \setbeamercolor{block body example}{bg=black!9,fg=black}
    \def\inserttheoremblockenv{exampleblock}
  }
\makeatother
\theoremstyle{mystyle}
\newtheorem*{defi}{Definition}

\setbeamertemplate{defis}[numbered]

% New command to display two-part functions 
%--------------------------------------------------------------  
\newcommand{\twopartdef}[4] 
{
	\left\{
		\begin{array}{ll}
			#1 & \mbox{if } #2 
			\\ \\
			#3 & \mbox{if } #4
		\end{array}
	\right.
}

% New enviroment without headline on the frame header
%--------------------------------------------------------------  
\makeatletter 
\newenvironment{noheadline}{
\setbeamertemplate{headline}{}
\addtobeamertemplate{frametitle}{\vspace*{-0.9\baselineskip}}{}
}{}
\makeatother

% To highlight/color specific rows in tables
%--------------------------------------------------------------  
\usepackage[beamer,customcolors]{hf-tikz}

\tikzset{hl/.style={
    set fill color=red!80!black!40,
    set border color=red!80!black,
  },
}

% Create the color for the verbatim background
%--------------------------------------------------------------  
\definecolor{cverbbg}{gray}{0.93}

% New environment for verbatin with background color
%--------------------------------------------------------------  
\newenvironment{cverbatim}
 {\SaveVerbatim{cverb}}
 {\endSaveVerbatim
  \flushleft\fboxrule=0pt\fboxsep=.5em
  \colorbox{cverbbg}{%
    \makebox[\dimexpr\linewidth-2\fboxsep][l]{\BUseVerbatim{cverb}}%
  }
  \endflushleft
}

% Get rid of the "references" section in the header
%------------------------------------------------------------
\makeatletter
\let\beamer@writeslidentry@miniframeson=\beamer@writeslidentry%
\def\beamer@writeslidentry@miniframesoff{%
  \expandafter\beamer@ifempty\expandafter{\beamer@framestartpage}{}% does not happen normally
  {%else
    % removed \addtocontents commands
    \clearpage\beamer@notesactions%
  }
}
\newcommand*{\miniframeson}{\let\beamer@writeslidentry=\beamer@writeslidentry@miniframeson}
\newcommand*{\miniframesoff}{\let\beamer@writeslidentry=\beamer@writeslidentry@miniframesoff}
\makeatother

% Add vertical space between the items of the TOC
%------------------------------------------------------------
\makeatletter
\patchcmd{\beamer@sectionintoc}
  {\vfill}
  {\vskip\itemsep}
  {}
  {}
\makeatother

% Bibliography packages and options
%------------------------------------------------------------
\usepackage[
	style=authoryear,
 	bibencoding=utf8, 
 	minnames=1, 
 	maxnames=3,
	maxbibnames=3, 
	backref=false, 
	natbib=true, 
	dashed=false, 
	terseinits=true, 
	giveninits=true, 
	uniquename=false, 
	uniquelist=true, 
	labeldateparts=true,
	doi=false, 
	isbn=false, 
	eprint = false,        % I disabled this. check if its appropriate
	url = false,             % I disabled this. check if its appropriate
natbib=true, backend=biber]{biblatex}

\DefineBibliographyStrings{english}{
    backrefpage = {Cited on page},
    backrefpages = {Cited on pages},
}

% Change the default bibliography formatting to be more "statistical"
%------------------------------------------------------------
\DeclareFieldFormat{url}{\url{#1}}
\DeclareFieldFormat[article]{pages}{#1}
\DeclareFieldFormat[inproceedings]{pages}{\lowercase{pp.}#1}
\DeclareFieldFormat[incollection]{pages}{\lowercase{pp.}#1}
\DeclareFieldFormat[article]{volume}{\mkbibbold{#1}}
\DeclareFieldFormat[article]{number}{\mkbibparens{#1}}
\DeclareFieldFormat[article]{title}{\MakeCapital{#1}}
\DeclareFieldFormat[article]{url}{}
\DeclareFieldFormat[book]{url}{}
\DeclareFieldFormat[inbook]{url}{}
\DeclareFieldFormat[thesis]{url}{}
\DeclareFieldFormat[thesis]{url}{}
\DeclareFieldFormat[incollection]{url}{}
\DeclareFieldFormat[inproceedings]{url}{}
\DeclareFieldFormat[inproceedings]{title}{#1}
\DeclareFieldFormat{shorthandwidth}{#1}
\DeclareFieldFormat[thesis]{citetitle}{#1}
\DeclareFieldFormat[thesis]{title}{#1} 

% No dot before number of articles
%------------------------------------------------------------
\usepackage{xpatch}
\xpatchbibmacro{volume+number+eid}{\setunit*{\adddot}}{}{}{}

% Remove In: for an article.
%------------------------------------------------------------
\renewbibmacro{in:}{%
  \ifentrytype{article}{}{%
  \printtext{\bibstring{in}\intitlepunct}}}
  
% Get rid of notes in citations
%------------------------------------------------------------
\AtEveryBibitem{\clearfield{note}}

% Get rid of months in citations
%------------------------------------------------------------
\AtEveryBibitem{\clearfield{month}}
\AtEveryCitekey{\clearfield{month}}
\raggedbottom

% Font for references
%------------------------------------------------------------
\renewcommand\bibfont{\scriptsize}

% If you have more than one page of references, you want to tell beamer
% to put the continuation section label from the second slide onwards
%------------------------------------------------------------
\setbeamertemplate{frametitle continuation}[from second]

% Get rid of all the colours
%------------------------------------------------------------
\setbeamercolor*{bibliography entry title}{fg=black}
\setbeamercolor*{bibliography entry author}{fg=black}
\setbeamercolor*{bibliography entry location}{fg=black}
\setbeamercolor*{bibliography entry note}{fg=black}

% No icon in references listing
%------------------------------------------------------------
\setbeamertemplate{bibliography item}{}

% path to the .bib file
%------------------------------------------------------------
\addbibresource{files/references.bib}



% Title and subtitle
%--------------------------------------------------------------
\title[Presentation with LaTeX]{Presentation with \LaTeX}
%\subtitle{A user friendly template}

% Your name
%--------------------------------------------------------------
\author[Pedro Fonseca]{\textbf {Pedro Fonseca}}

% Date
%--------------------------------------------------------------
\date{\today}

% Beginning of the document
%--------------------------------------------------------------
\begin{document}

% Title page
%--------------------------------------------------------------

% Title page
%--------------------------------------------------------------

\begin{frame}
	\titlepage
\end{frame}



%--------------------------------------------------------------
%--------------------------------------------------------------
%  This is where the frames begin
%--------------------------------------------------------------
%--------------------------------------------------------------

% Frame
%--------------------------------------------------------------

\begin{frame}{The first frame}
	\begin{itemize}
		\item This is a simple beamer template with ISEG's colors
		\item In the next slides wou will see some examples of what you can do with beamer.
		\item Inspect the underlying code to learn how to make your own frames.
	\end{itemize}
\end{frame}

% Frame
%--------------------------------------------------------------

\begin{frame}{Lists of items}
	In this presentation you will learn how to:
	\begin{itemize}
	\item make ordered and unordered lists of items
	\item include pictures
	\item use mathematical formulas
	\item display code
	\item create theorems and definitions
	\end{itemize}
\end{frame}

% Frame
%--------------------------------------------------------------

\begin{frame}{Enumeration of items}
	In the last frame you learned how to create a list of items. Now lets create an enumeration of items:
	\begin{enumerate}
	\item make ordered and unordered lists of items
	\item include pictures
	\item use mathematical formulas
	\item display code
	\item create theorems and definitions
	\end{enumerate}
\end{frame}

% Frame
%--------------------------------------------------------------

\begin{frame}{Itemized lists inside itemized lists}
	You can have itemized lists inside itemized lists:
	\begin{itemize}
	\item lists of items
		\begin{itemize}
		\item ordered
		\item unordered
		\end{itemize}
	\item include pictures
	\item use mathematical formulas
	\item display code
		\begin{itemize}
		\item Python code
		\item R code
		\end{itemize}
	\item create theorems and definitions
	\end{itemize}
\end{frame}

% Frame
%--------------------------------------------------------------

\begin{frame}{Including pictures}
	This picture is in the ``figures'' folder. We also included a caption:
	\begin{figure}
	\includegraphics[scale=.5]{figures/graph.png}
	\caption{Fuel consumption (Hwy) vs engine size (displ)}
	\end{figure}
\end{frame}

% Frame
%--------------------------------------------------------------

\begin{frame}{Definitions and theorems}
	You can display definitions and theorems:
	\begin{definition}[Fibration]
	A fibration is a mapping between two topological spaces that has the homotopy lifting property for every space $X$.
	\end{definition}
	Latex is very useful to write mathematical expressions:
	\begin{theorem}[Bayes]
	$P(\theta|\textbf{D}) = P(\theta ) \frac{P(\textbf{D} |\theta)}{P(\textbf{D})}$
	\end{theorem}
\end{frame}

% Frame
%--------------------------------------------------------------
\begin{frame}{The block environment}
	The block environment is more versatile. You can write whatever you want on the red block:
	\begin{block}{Pythagorean theorem}
	This is a theorem about right triangles and can be summarised in the next 
	equation 
	\[ x^2 + y^2 = z^2 \]
	\end{block}
	\begin{block}{Corollary}
	There's no right rectangle whose sides measure 3cm, 4cm, and 6cm.
	\end{block}
\end{frame}

% Frame
%--------------------------------------------------------------

\begin{frame}{Two part functions}
	I also included an environment that allows you to easily write two part functions:
	$$|x| =\twopartdef{x}{x \geq 0} {-x} {x < 0}$$
\end{frame}

% Frame
%--------------------------------------------------------------

\begin{frame}{Examples of mathematical expressions}
	\begin{itemize}
	\item Likelihood: $\bm{x}|\bm{\theta} \sim \mathcal{M}_k(N,\bm{\theta})$
	\item Hypotheses:  $ H_0: \bm{\theta}=\bm{\theta_0} \,\,\,\, \text{vs} \,\,\,\,  H_1: \bm{\theta} \neq \bm{\theta_0}$
	\item Prior: $\pi(\bm{\theta}|H_0)={1_{\bm{\theta_0}}(\bm{\theta_0})}$ and $\bm{\theta}|H_1, \bm{\alpha} \sim \operatorname{Dir}_k(\bm{\alpha})$ 
	\item Marginal density: $m_i(\bm{x})=\int_{\Theta_i} f(\bm{x}|\bm{\theta})\pi(\bm{\theta}|H_i) \, d\bm{\theta} $, \, $i=0,1$
		\begin{itemize}
		\item  $m_0(\bm{x})=\frac{N!}{\prod_{i=1}^{k+1}x_i!}\prod_{i=1}^{k+1}{\theta_{0i}}^{x_i}$ 
		\item  $m_1(\bm{x})= \frac{N!}{\prod_{i=1}^{k+1}x_i!}\frac{B(\bm{\alpha}+\bm{x})}{B(\bm{\alpha})}$ 
		\end{itemize}
	\item Bayes factor: $B_{01}(\bm{x}) = \frac{m_0(\bm{x})}{m_1(\bm{x})} = \frac{\prod_{i=1}^{k+1}{(\theta_{0i}}^{x_i}) \prod_{i=1}^{k+1}[\Gamma(\alpha_i)] \Gamma [\sum_{i=1}^{k+1}(\alpha_i+x_i)]}{\Gamma(\sum_{i=1}^{k+1} \alpha_i) \prod_{i=1}^{k+1}\Gamma(\alpha_i+x_i)}$
	\end{itemize}
\end{frame}

% Frame
%--------------------------------------------------------------

\begin{frame}[fragile]{Python code looks good on verbatim}
	Code is usually displayed in verbatim. This code prints the Fibonacci sequence:
	\begin{verbatim}
	nterms = int(input("How many terms? "))
	n1, n2 = 0, 1
	count = 0
	print("Fibonacci sequence:")
	while count < nterms:
	    print(n1)
	    nth = n1 + n2
	    n1 = n2
	    n2 = nth
	    count += 1
	}
	\end{verbatim}
\end{frame}

% Frame
%--------------------------------------------------------------

\begin{frame}[fragile]{R code looks good too}
	I also included a verbatim environment with background color:
	\begin{cverbatim}
getmode <- function(v) {
	uniqv <- unique(v)
	uniqv[which.max(tabulate(match(v, uniqv)))]
	}
	\end{cverbatim}
	With this R code you can build a function that calculates the mode.
\end{frame}

% End of the document
%--------------------------------------------------------------
\end{document}















